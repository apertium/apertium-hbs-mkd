

\documentclass{book}

\usepackage[DRAFT]{cslipubs}

\usepackage[T2A]{fontenc}
\usepackage[utf8]{inputenc}

\usepackage{footnote}
\usepackage{multibib}

\title{A rule-based machine translation from Serbo-Croatian to Macedonian}
%\author{Hrvoje Peradin, Francis Tyers}

%TODO: Ubit ovo naknadno, jer se ne bi smjelo pojaviti u člančetu
%\usepackage{color}
%\newcommand{\todo}[1]{\textcolor{red}{#1}}

\newcommand{\todo}[1]{\textbf{TODO: #1}}


\hfuzz = 2pt

\usepackage[comma]{natbib}

\usepackage{chapterbib}

\usepackage{import}         % Allows chapters to have separate subdirectories.

\usepackage{makeidx}        % Necessary if this book has an Index.
\makeindex                  % Comment out this line when Index is final.

\usepackage{cslipubs-extra}

% Some more bells and whistles.
%\usepackage{avm}
\usepackage{lingmacros}

\begin{document}

\frontmatter      %%%%%% Start with Roman page numbers and unnumbered chapters

\maketitle

\setcounter{page}{5}
\setcounter{tocdepth}{0}  % 0 = chapters only in Contents, 1 = sections also.
%\tableofcontents          % LaTeX at least twice for correct Table of Contents.

\chapter{Introduction}

%\begin{abstract}
This paper describes the development of a one-way machine translation
from Serbo-Croatian to Macedonian in the Apertium platform.
Details of resources and development methods are given, as well as an
evaluation, and general directives for future work.
%\end{abstract}

The modern Macedonian language was standardised in 1944. and
is the official language of the Republic of Macedonia.

Serbo-Croatian is a term that encompasses four standard languages 
(Bosnian, Croatian, Montenegrin and Serbian) based on the 
\emph{neoshtokavian} dialect. The standardisation of the language started
in the 19th century, as an attempt to unify the literary and linguistic 
traditions of the south Slavic area. The standard remained pluricentric
until the dissolution of Yugoslavia. Due to the large similarities between 
the standards we have decided to group them into one module, with a common
mode for analysis. Having in mind future work we have added separate modes 
for generation of Bosnian, Croatian and Serbian.\footnote{The current standardisation
process of Montenegrin is under way, and we are awaiting the outcome to implement a separate
mode}

Serbo-Croatian and Macedonian are largely mutually intelligible, however despite their close relation the differences in morphology 
%(notably the absence of noun declension in Macedonian vs. a typically Slavic rich declension in Serbo-Croatian) 
create difficulties in translation. For this reason the system is currently mono-directional (sh$\rightarrow$mk). 
The direction was chosen since it is easier to construct a system translating from a more detailed morphology 
(e.g. Macedonian "во куќа" can be translated both as "u kući" [in the house.LOC] and "u kuću"[into the house.ACC]).

Other systems currently supporting the languages are notably Google Translate\footnote{Supports Croatian, Serbian and Macedonian} and Systran.\footnote{Language pairs Serbian$\rightarrow$English, Croatian$\rightarrow$English}

\mainmatter

\chapter{The system}
\section{Design}
\subsection* {The Apertium platform}
\nocite{forcada2011apertium}
The platform follows a modular machine translation model.
Morphological analysis of the source text is performed by
letter transducers, and cohorts\footnote{A cohort consists of a surface form and one or more readings containing the lemma of the word and the morphological analysis.} obtained in this manner go through
a disambiguation process. 
Disambiguated readings proceed to a bilingual dictionary also performed 
by a letter transducer and  through a two-level syntactic transfer, which performs word reordering, deletions, insertions, and basic syntactic chunking.
The final module is also a letter transducer which generates surface forms 
in the target language from the bilingual transfer output.

\subsection* {Constraint Grammar}
The 
%initial 
disambiguation in this language pair is performed by a
Constraint Grammar (CG) module. CG is a 
paradigm that uses hand-written rules to reduce the problem 
of linguistic ambiguity. A series of context-dependent rules are applied 
to a stream of tokens and readings for a given surface form are excluded, 
selected or assigned additional tags.

\section{Development}
\subsection*{Resources}
Although some resources for morphological analysis of Serbian and Croatian exist
(\citealp{vitas2004intex}, \citealp{vitas2003processing}, \citealp{agic2008improving}, \citealp{snajder08automatic}), 
to our knowledge there are none freely available for either Serbian, Bosnian or 
Croatian. Thus the monolingual dictionary for Serbo-Croatian has been developed 
almost from scratch, with the aid of a Croatian grammar (\citealp{baric1997hrvatska}), 
and on-line resources such as \emph{Hrvatski jezični portal},\footnote{http://hjp.srce.hr}
wiktionaries and Wikipedia, as well as an SETimes corpus (\citealp{tyers2010south}) and a 
corpus composed from the Serbian, Bosnian, Croatian and Serbo-Croatian Wikipedias.

Bilingual resources available were also scarce. We used a parallel corpus obtained from
SETimes, and a Serbian--Macedonian on-line dictionary.\footnote{http://rechnik.on.net.mk/}

The morphological analyser/generator for Macedonian was taken from
\texttt{apertium-mk-bg} (\citealp{rangelov2011rule}) which is freely available under Apertium.
For reference on the Macedonian language we used the SEELRC reference grammar\footnote{http://slaviccenters.duke.edu/projects/grammars} and
Дигитален речник на македонскиот јазик.\footnote{http://www.makedonski.info/}

\subsection*{Analysis and generation}
The morphological analyser for Serbo-Croatian was written in the XML formalism of
\emph{lttoolbox} (\citealp{rojas2005construccion}), almost entirely from scratch, with the
aim to match the lexicon of the analyser from \texttt{apertium-mk-bg}. Since we intended to create a resource for all three
standards, a paradigm was assigned
to the reflex of the vowel yat\footnote{Typically in ekavian it is either a long or short "e", while in ijekavian the long variant is reflected as "ije", and the short as "je"} to enable analysis of both ekavian and ijekavian dialects, 
and the extended metadix format was used to enable separating different standards by
analysis and generation modes.

The basic inflectional paradigms were taken from the Croatian grammar (\citealp{baric1997hrvatska}), 
and further refined according to new entries (e.g. with voice changes not covered by basic declension
patterns).

The entries were made mostly manually, with some proper nouns obtained semi-automatically from the Macedonian
dictionary.

\subsection*{Disambiguation}
The disambiguation was done by a Constraint Grammar module.
In case of remaining ambiguity, the system picks the first analysis from the output
of the disambiguation module.

%a combination of an HMM tagger (\texttt{apertium-tagger}, trained on a Wikipedia corpus)
%and 

%Though mature CG systems have proven to have a high Precision / Recall score
%(up to 99\%, \citealp{bick2000parsing}), the CG for \texttt{apertium-sh-mk}
%has been developed in a limited time frame, and has relatively few rules (170).
%Thus it has been used only as a pre-disambiguator for the HMM tagger.

The following are examples of disambiguation rules:
\begin{itemize}
\item Preposition-based case disambiguation:\nopagebreak

\texttt{REMOVE Preposition + \$\$Case IF (1 Nominal - \$\$Case)}

\texttt{REMOVE Nominal + \$\$Case IF (NOT -1 Preposition + \$\$Case) (NOT -1 Modifier + \$\$Case)}

The first rule cleans the grammatical case from a preposition\footnote{The cases the prepositions govern
are marked on the analyses of the prepositions.} 
if it is not followed by a noun, pronoun or adjective in the same case. The second
rule, similarly, cleans the case from a noun, pronoun or adjective if
it is not preceded directly either by a preposition which governs the case
or a modifier (e.g. adjective or demonstrative pronoun) in the same case.
\footnote{The \$\$ prefix signifies unification, i.e. iteration over the set of all
grammatical cases.}

\item Noun phrases:

\texttt{REMOVE Modifier + \$\$GenNum IF (1 Nominal - \$\$GenNum)}

\texttt{REMOVE Nominal + \$\$GenNum IF (-1 Modifier - \$\$GenNum)}

These rules operate on noun phrases, and use the gender and number agreement
to eliminate grammatically impossible readings. For example, 
if a neuter/feminine ambiguous adjective precedes a feminine noun, the neuter
reading is removed. Similarly, if a singular/plural ambiguous noun is preceded
by an unambiguously plural adjective, the singular reading is removed.

\item Adverb / adjective ambiguity:

\texttt{SELECT Adverb IF (0 Adverb OR Adjective) (1 Verb)}

This simple rule resolves a common ambiguity by selecting the adverb reading if the word is followed by a verb.

\item Dative / locative ambiguity:

\texttt{SELECT Dative IF (0 Dative OR Locative) (NOT -1 Preposition) (NOT -1 Modifier + Locative)}

The cases are orthographically identical, however locative
is purely prepositional, so in most cases the ambiguity is easily
resolved by selecting dative if the phrase
is not preceded by a locative preposition.

\end{itemize}

\subsection*{Lexical transfer}
The bilingual lexicon was written using the \emph{lttoolbox} format, and composed mostly 
manually, with paradigms added to compensate the tag set differences.
Translation entries were added according to the lexicon from 
the Macedonian analyser. Having in mind future work, translations specific solely 
to Bosnian, Croatian or Serbian standard were grouped in respective sections.

\subsection*{Syntactic transfer}
Despite the close relation of the two languages, there are substantial 
differences in morphology, and structures with analogous functionality 
are not necessarily morphologically cognate. Therefore we have used 
a two level syntactic transfer. 

The first level performs tag mappings, normalisation (e.g. 
case to nominative, infinitive to present), 
rudimentary transformations, and packing of phrases in syntactic chunks. 

Examples of transfer rules:

\begin{itemize}
\item The future tense:
\enumsentence{

Ja ću gledati\footnote{The encliticized future tense forms (gledat ću / gledaću) are handled equally} $\rightarrow$ Јаз ќе гледам

I will.P1.SG watch $\rightarrow$ I will watch.P1.SG
}

Serbo-Croatian uses a clitic + infinitive form with a declinable clitic, while
Macedonian uses a frozen clitic form, and the person/number is marked on the verb.
Thus several rules were written to match occurrences of future tense
and transfer the information in translation.

\item Clitic reordering:

\enumsentence{

...okrenut ću se... $\rightarrow$ ...ќе се обрнам...

...[turn] [I will] [myself]... $\rightarrow$ ...[I will] [myself] [turn]...

(I will turn myself around)
}

The order of clitics in both languages is different, so a series of rules was written to rearrange them.

\item Cases as prepositional phrases:

\enumsentence{

Let avionom\footnote{The Croatian normative 'zrakoplov' is also accepted and translated equally} $\rightarrow$ Летање со авион.

[flight] [by aeroplane.INS] $\rightarrow$ [flight] [with] [aeroplane]

(Flying by an aeroplane.)
}

While Serbo-Croatian has seven morphological cases, Macedonian has 
completely replaced its declension system with analytic, prepositional and clitic constructions.
The second level of transfer replaces simple noun and adjective phrases with prepositional constructions.

\item Inference of definiteness:

\enumsentence{

U sastavu Vojske Srbije $\rightarrow$ Во составот на Српската војска\footnote{The article in Macedonian
attaches to the first constituent of the noun phrase}

[in] [composition] [of Serbian Army] $\rightarrow$ [In] [composition.DEF] [of Serbian Army]

(In the composition of the Serbian Army)

}

The definite article in Macedonian has no analogy in Serbo-Croatian (except to some extent the definiteness of adjectives). This transfer rule infers definiteness for a common noun preceding
a proper noun in genitive.

\end{itemize}

\subsection*{Status}
The current status of the language pair is given in Table \ref{tab:status}.

\begin{table}
\begin{center}
\caption{Status of apertium-sh-mk as of April 11 2011.}
\label{tab:status}
\begin{tabular}{l|r}
Module & Entries / Rules \\
\hline
Serbo-Croatian dictionary & 7564 \\
Macedonian dictionary & 8672 \\
Bilingual dictionary & 9985 (unique) \\
Transfer rules (1 and 2) & 51 + 11 \\
Serbo-Croatian CG & 170 \\
\hline
\end{tabular}
\end{center}
\end{table}

\section{Evaluation}

This section presents an evaluation of the system performance, with
coverage measured on two corpora, and a quantitative analysis.

\subsection*{Coverage}

The data for coverage of the Serbo-Croatian analyser is 
given in Table \ref{tab:coverage}. Coverage is naive, 
it means that for any given form in the source language at least one analysis
has been given. The analyser has been tested on a combined Wikipedia
corpus, and on a corpus of Serbian and Croatian
SETimes articles. The corpora was divided in four parts and average 
coverage calculated.

\begin{table}
\begin{center}
\caption{Coverage}
\label{tab:coverage}
\begin{tabular}{l|c|c}
Corpus & Coverage & Std. dev. \\
\hline
Wikipedia (sh+bs+sr+hr) & 73.12\% & 0.36 \\
SETimes (sr + hr) & 82.64\% & 0.38 \\
\hline
\end{tabular}
\end{center}
\end{table}

\subsection*{Quantitative evaluation}

Quantitative evaluation has been performed on four articles from SETimes. The articles
were translated by Apertium, and post-edited by a human translator. The word error rate (WER)
and the position-independent error rate (PER) were calculated by the number of changes the
human editor needed to make. Results are given in Table \ref{tab:quantitative}.

\begin{savenotes}
\begin{table}
\begin{center}
\begin{tabular}{l|c|c|c|c}
Article & OOV\footnote{Out of vocabulary words} & Words & WER & PER \\
\hline
\texttt{setimes.pilots.txt} & 0.44\% & 454 & 29.96\% & 20.48\% \\
\texttt{setimes.tablice.txt} & 0.43\% & 470 & 48.12\% & 34.58\% \\
\texttt{setimes.klupa.txt} & 18.12\% & 480 & 60.36\% & 46.81\% \\
\texttt{setimes.povijest.txt} & 14.18\% & 529 & 53.35\% & 40.52\% \\
\hline
\end{tabular}
\footnotetext{Out of vocabulary words}
\caption{Quantitative evaluation}
\label{tab:quantitative}
\end{center}
\end{table}
\end{savenotes}



\subsection*{Common problems}

Although CG rules successfully rule out quite a lot of
grammatically impossible analyses, the number of rules for this language pair
is quite low, so disambiguation is not always correct.

Another obvious source of errors are unknown words, which typically disrupt 
the flow of disambiguation, especially when they occur inside noun phrases.

The definite article is quite difficult to infer. Though in limited
cases it can be transferred from definite adjectives, or guessed from
specific context, there is e.g. no straightforward way to mark a subject previously
introduced in discourse as definite.

Serbo-Croatian cases do not translate consistently to prepositional constructions.
A notable example is the partitive vs. possessive genitive. The phrase "čaša vode"
can be translated as "чаша вода" ("a glass of water") or "чашата на вода" 
(the water's glass).

Both languages have a very free word order. For instance, an 
adjective can modify a noun arbitrarily far to the left
(\emph{Vožnja}.N.FEM zrakoplovom ... bila je \emph{odlučujuća}.ADJ.FEM $\rightarrow$ \emph{Возенје}.N.NEUT со авионот ... беше \emph{решавачка}.ADJ.FEM  [The airplane \emph{ride} ... was \emph{decisive}])
If the noun 
changes gender in translation, the adjective is not matched to it, and retains
the source language gender.

\section{Discussion}
This paper presented the design and an evaluation of a language pair for the Apertium platform.
It is the first rule-based MT system between Serbo-Croatian\footnote{It is to our knowledge the first 
MT system supporting Bosnian} and Macedonian, and the morphological
analyser and CG module are currently only such open-source resources for Serbo-Croatian.

The system was dubbed by a native speaker as "overall fine", there are obvious errors, but the output is legible and easily post-editable. 

A significant part of the problems is typical for a system in such an early phase of 
development. The morphological lexicons for both languages are small, and 
the same remark can be made for the number of disambiguation rules. 

Some ambiguities that arise in analysis of the source language are 
difficult or impossible to resolve in a simple rule-based manner, which suggests that 
the system should be combined with machine learning and statistical methods.

In terms of future work the essential task is to increase coverage, to enable working
with larger corpora, and to improve the disambiguation rules, which make a significant
contribution to translation quality.

\section*{Acknowledgements}
The development of this language pair was funded as a part of the 
Google Summer of Code.\footnote{http://code.google.com/soc/}
Many thanks to my GSoC mentor Francis Tyers, for his help and inexhaustible
patience, to Jimmy O'Regan, Kevin Unhammer and other Apertium contributors for
their help on Apertium, to Tino Didriksen for his help on CG, to Tihomir Rangelov
for his advice in the initial phase of this work, and to Dime Mitrevski
for his input on the Macedonian language.

\bibliographystyle {cslipubscollection/cslipubs-natbib}
\bibliography {apertium-sh-mk}

\backmatter
\end{document}
